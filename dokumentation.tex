%%**************************************************************
%% Vorlage fuer Bachelorarbeiten (o.ä.) der DHBW
%%
%% Autor: Tobias Dreher, Yves Fischer
%% Datum: 06.07.2011
%%**************************************************************

\newcommand{\pdftitel}{Title of the document}
\newcommand{\autor}{Max Mustermann}
\newcommand{\arbeit}{Bachelor thesis}

\input{header}

% Ab jetzt können auch Umlaute verwendet werden
\newcommand{\titel}{Usually the title of a Bachelor thesis has a length of two lines}
\newcommand{\matrikelnr}{1234567}
\newcommand{\kurs}{ABC2008DE}
\newcommand{\datumAbgabe}{August 2011}
\newcommand{\firma}{Firma GmbH}
\newcommand{\firmenort}{Firmenort}
\newcommand{\abgabeort}{Abgabeort}
\newcommand{\abschluss}{Bachelor of Arts}
\newcommand{\studiengang}{Vorderasiatische Archäologie}
\newcommand{\dhbw}{Stuttgart Campus Horb}
\newcommand{\betreuer}{Dipl.-Ing. (FH) Peter Pan}
\newcommand{\gutachter}{Dr. Silvana Koch-Mehrin}
\newcommand{\zeitraum}{12 Weeks}

\makeglossaries
\input{glossary}

\begin{document}

	% Deckblatt
	\begin{spacing}{1}
		\input{deckblatt}
	\end{spacing}
	\newpage
	
	\renewcommand{\thepage}{\Roman{page}}
	\setcounter{page}{1}
	
	% Erklärung
	\input{erklaerung}
	\newpage

	% Abstract
	\input{abstract}
	\newpage

	\pagestyle{plain}

	% Inhaltsverzeichnis
	\begin{spacing}{1.1}
		\setcounter{tocdepth}{3}
		\tableofcontents
	\end{spacing}
	\newpage

	\renewcommand{\thepage}{\arabic{page}}
	\setcounter{page}{1}

	% Inhalt
	\input{content/01kapitel}
	\input{content/02kapitel}

	% Anhang
	\clearpage
	\pagenumbering{roman}

	% Abbildungsverzeichnis
	\cleardoublepage
	\phantomsection \label{listoffig}
	\addcontentsline{toc}{chapter}{List of figures}
	\listoffigures

	%Tabellenverzeichnis
	\cleardoublepage
	\phantomsection \label{listoftab}
	\addcontentsline{toc}{chapter}{List of tables}
	\listoftables

	% Quellcodeverzeichnis
	\cleardoublepage
	\phantomsection \label{listoflist}
	\addcontentsline{toc}{chapter}{Listings}
	\lstlistoflistings

	% Literaturverzeichnis
	\cleardoublepage
	\phantomsection \label{listoflit}
	\addcontentsline{toc}{chapter}{Bibliography}
	\input{literatur}

	% Abkürzungsverzeichnis
	% vorher in Konsole folgendes aufrufen: 
	%	makeglossaries makeglossaries dokumentation.acn && makeglossaries dokumentation.glo
	\printglossary[type=\acronymtype]

	% Glossar
	\printglossary[style=altlist,title=Glossary]
\end{document}
