%%**************************************************************
%% Vorlage fuer Bachelorarbeiten (o.ä.) der DHBW
%%
%% Autor: Tobias Dreher, Yves Fischer
%% Datum: 06.07.2011
%%**************************************************************

\newcommand{\pdftitel}{Title of the document}
\newcommand{\autor}{Max Mustermann}
\newcommand{\arbeit}{Bachelor thesis}

\input{header}

% Ab jetzt können auch Umlaute verwendet werden
\newcommand{\titel}{Usually the title of a Bachelor thesis has a length of two lines}
\newcommand{\matrikelnr}{1234567}
\newcommand{\kurs}{ABC2008DE}
\newcommand{\datumAbgabe}{August 2011}
\newcommand{\firma}{Firma GmbH}
\newcommand{\firmenort}{Firmenort}
\newcommand{\abgabeort}{Abgabeort}
\newcommand{\abschluss}{Bachelor of Arts}
\newcommand{\studiengang}{Vorderasiatische Archäologie}
\newcommand{\dhbw}{Stuttgart Campus Horb}
\newcommand{\betreuer}{Dipl.-Ing. (FH) Peter Pan}
\newcommand{\gutachter}{Dr. Silvana Koch-Mehrin}
\newcommand{\zeitraum}{12 Weeks}

\makeglossaries
\input{glossary}

\begin{document}

	% Deckblatt
	\begin{spacing}{1}
		\begin{titlepage}
	\begin{longtable}{p{.55\textwidth} p{.85\textwidth}}
	  {\includegraphics[height=2.6cm]{images/logo.png}} & 
	  {\includegraphics[height=2.6cm]{images/dhbw.png}}
	\end{longtable}
	\enlargethispage{20mm}
	\begin{center}
	  \vspace*{12mm}	{\LARGE\bf \titel }\\
	  \vspace*{12mm}	{\large\bf \arbeit}\\
	  \vspace*{12mm}	for the\\
	  \vspace*{3mm} 	{\bf \abschluss}\\
	  \vspace*{12mm}	at Course of Studies \studiengang\\
	  \vspace*{3mm} 	at the Cooperative State University \dhbw\\
	  \vspace*{12mm}	by\\
	  \vspace*{3mm} 	{\large\bf \autor}\\
	  \vspace*{12mm}	\datumAbgabe\\
	\end{center}
	\vfill
	\begin{spacing}{1.2}
	\begin{tabbing}
		mmmmmmmmmmmmmmmmmmmmmmmmmm     \= \kill
		\textbf{Time of Project}  \>  \zeitraum\\
		\textbf{Student ID, Course}  \>  \matrikelnr, \kurs\\
		\textbf{Company}      \>  \firma, \firmenort\\
		\textbf{Supervisor in the Company}              \>  \betreuer\\
		\textbf{Reviewer}             \>  \gutachter
	\end{tabbing}
	\end{spacing}
\end{titlepage}

	\end{spacing}
	\newpage
	
	\renewcommand{\thepage}{\Roman{page}}
	\setcounter{page}{1}
	
	% Erklärung
	\thispagestyle{empty}

\section*{Author's declaration}
% Seite 8
% http://studium.ba-bw.de/fileadmin/media/allgemein/bestimmungen/btechnik/richtlinien/Richtlinien_Praxismodule_Studien_und_Bachelorarbeiten_2011.pdf
\vspace*{2em}
Unless otherwise indicated in the text or references, or acknowledged above, this thesis is entirely the product of my own scholarly work. This thesis has not been submitted either in whole or part, for a degree at this or any other university or institution. This is to certify that the printed version is equivalent to the submitted electronic one.
\vspace{3em}

\abgabeort, \datumAbgabe
\vspace{4em}

\rule{6cm}{0.4pt}\\
\autor



	\newpage

	% Abstract
	\input{abstract}
	\newpage

	\pagestyle{plain}

	% Inhaltsverzeichnis
	\begin{spacing}{1.1}
		\setcounter{tocdepth}{3}
		\tableofcontents
	\end{spacing}
	\newpage

	\renewcommand{\thepage}{\arabic{page}}
	\setcounter{page}{1}

	% Inhalt
	\input{content/01kapitel}
	\input{content/02kapitel}

	% Anhang
	\clearpage
	\pagenumbering{roman}

	% Abbildungsverzeichnis
	\cleardoublepage
	\phantomsection \label{listoffig}
	\addcontentsline{toc}{chapter}{List of figures}
	\listoffigures

	%Tabellenverzeichnis
	\cleardoublepage
	\phantomsection \label{listoftab}
	\addcontentsline{toc}{chapter}{List of tables}
	\listoftables

	% Quellcodeverzeichnis
	\cleardoublepage
	\phantomsection \label{listoflist}
	\addcontentsline{toc}{chapter}{Listings}
	\lstlistoflistings

	% Literaturverzeichnis
	\cleardoublepage
	\phantomsection \label{listoflit}
	\addcontentsline{toc}{chapter}{Bibliography}
	\input{literatur}

	% Abkürzungsverzeichnis
	% vorher in Konsole folgendes aufrufen: 
	%	makeglossaries makeglossaries dokumentation.acn && makeglossaries dokumentation.glo
	\printglossary[type=\acronymtype]

	% Glossar
	\printglossary[style=altlist,title=Glossary]
\end{document}
